\documentclass[9pt]{beamer}
\usetheme{cmepda}

\usepackage[utf8]{inputenc}
\usepackage[T1]{fontenc}


\title{Python Basics (2/2)}
\subtitle{Computing Methods for Experimental Physics and Data Analysis}
\date{Compiled on \today}
\author{L. Baldini}
\institute[UNIPI and INFN]{Universit\`a and INFN--Pisa}
\email{luca.baldini@pi.infn.it}


\begin{document}


\titleframe

\begin{frame}
  \frametitle{What is the Python standard library?}
  \begin{itemize}
  \item Three levels hierarchy:
    \begin{itemize}
    \item The Python core language (all you get at the interpreter startup)
    \item The Python standard library (e.g., \texttt{math})
    \item An enourmous number of third-party packages (e.g., \texttt{numpy})
    \end{itemize}
  \item The standard library is included in every Python distribution
    \begin{itemize}
    \item And it is (slowly) evolving with time
    \end{itemize}
  \item With third-party packages you are on your own
    \begin{itemize}
    \item Although Anaconda solves many of the issues
    \item And if you are using GNU-Linux your package manager is probably
      taking care of everything for you
    \end{itemize}
  \item (Well---and of course there are your own modules, too\ldots)
  \item \alert{Anything that is out of the core is loaded in memory via an
    \texttt{import} statement}
  \end{itemize}
\end{frame}


\begin{frame}[fragile]
  \frametitle{Digression: the import system}
  \framesubtitle{Basics and best practices}
  \begin{Verbatim}
from math import *
[...]
# Terrible: where the hell is sqrt coming from?
x = sqrt(2.)

from math import sqrt
[...]
# Better: if you haven't redefined sqrt this is from the math library
x = sqrt(2.)

import math
[...]
# Best: five more characters, but at least is clear where sqrt is coming from
x = math.sqrt(2.)
  \end{Verbatim}
  \begin{itemize}
  \item The \texttt{\$PYTHONPATH} environmental variables is your friend
    to control where you want to import modules from
    \begin{itemize}
    \item You will need to tweak it when you start writing your own packages
    \end{itemize}
  \item You will need suitable \texttt{\_\_init\_\_.py} files to navigate
    directories
  \end{itemize}
\end{frame}


\begin{frame}
  \frametitle{More on the import system}
  \begin{itemize}
  \item The import system is fairly flexible
    \begin{itemize}
    \item Take advantage of it but don't abuse it
    \end{itemize}
  \item This is ok\ldots
  \end{itemize}

  \medskip
  \begin{Verbatim}[label=\makebox{\url{https://github.com/lucabaldini/cmepda/tree/master/slides/latex/snippets/import1.py}},commandchars=\\\{\}]
\PY{c+c1}{\PYZsh{} This is ok, and vastly recognized by the community}
\PY{k+kn}{import} \PY{n+nn}{numpy} \PY{k+kn}{as} \PY{n+nn}{np}
\PY{k+kn}{from} \PY{n+nn}{matplotlib} \PY{k+kn}{import} \PY{n}{pyplot} \PY{k}{as} \PY{n}{plt}

\PY{n}{x} \PY{o}{=} \PY{n}{np}\PY{o}{.}\PY{n}{linspace}\PY{p}{(}\PY{l+m+mf}{0.}\PY{p}{,} \PY{l+m+mf}{10.}\PY{p}{,} \PY{l+m+mi}{100}\PY{p}{)}
\PY{n}{y} \PY{o}{=} \PY{n}{x}\PY{o}{*}\PY{o}{*}\PY{l+m+mf}{2.}
\PY{n}{plt}\PY{o}{.}\PY{n}{plot}\PY{p}{(}\PY{n}{x}\PY{p}{,} \PY{n}{y}\PY{p}{)}
\end{Verbatim}

  \begin{itemize}
  \item \ldots and this is a catastrophe!
  \end{itemize}

  \medskip
  \begin{Verbatim}[label=\makebox{\url{https://bitbucket.org/lbaldini/programming/src/tip/snippets/import2.py}},commandchars=\\\{\}]
\PY{k+kn}{from} \PY{n+nn}{math} \PY{k+kn}{import} \PY{o}{*}
\PY{k+kn}{import} \PY{n+nn}{logging} \PY{k+kn}{as} \PY{n+nn}{log}

\PY{c+c1}{\PYZsh{} ... 1000 lines of code in the middle}

\PY{n}{x} \PY{o}{=} \PY{n}{log}\PY{p}{(}\PY{l+m+mf}{2.}\PY{p}{)}

[Output]
Traceback (most recent call last):
  File "snippets/import2.py", line 6, in <module>
    x = log(2.)
TypeError: 'module' object is not callable
\end{Verbatim}
\end{frame}


\begin{frame}
  \frametitle{Overview of the standard library}
  \framesubtitle{\texttt{time}, \texttt{datetime} and \texttt{calendar}}
  \begin{itemize}
  \item Collections of facilities related to date and time
    \begin{itemize}
    \item Measure the execution time of your scripts
    \item Convert from time to date and vice-versa
    \end{itemize}
  \item This is all but trivial!
    \begin{itemize}
    \item Ever heard of UNIX time? And UTC? And time zones?
    \end{itemize}
  \end{itemize}
\end{frame}


\begin{frame}[fragile]
  \frametitle{Overview of the standard library}
  \framesubtitle{\texttt{math}}
  \begin{Verbatim}
Python 3.7.4 (default, Jul  9 2019, 16:32:37) 
[GCC 9.1.1 20190503 (Red Hat 9.1.1-1)] on linux
Type "help", "copyright", "credits" or "license" for more information.
>>> import math
>>> dir(math)
['__doc__', '__file__', '__loader__', '__name__', '__package__', '__spec__',
'acos', 'acosh', 'asin', 'asinh', 'atan', 'atan2', 'atanh', 'ceil', 'copysign',
'cos', 'cosh', 'degrees', 'e', 'erf', 'erfc', 'exp', 'expm1', 'fabs',
'factorial', 'floor', 'fmod', 'frexp', 'fsum', 'gamma', 'gcd', 'hypot', 'inf',
'isclose', 'isfinite', 'isinf', 'isnan', 'ldexp', 'lgamma', 'log', 'log10',
'log1p', 'log2', 'modf', 'nan', 'pi', 'pow', 'radians', 'remainder', 'sin',
'sinh', 'sqrt', 'tan', 'tanh', 'tau', 'trunc']
>>>
  \end{Verbatim}

  \medskip

  \begin{itemize}
  \item If you work a lot with arrays you will end up using mostly
    \texttt{numpy}
  \end{itemize}
\end{frame}


\begin{frame}[fragile]
  \frametitle{Overview of the standard library}
  \framesubtitle{\texttt{random}}
  \begin{Verbatim}
Python 3.7.4 (default, Jul  9 2019, 16:32:37) 
[GCC 9.1.1 20190503 (Red Hat 9.1.1-1)] on linux
Type "help", "copyright", "credits" or "license" for more information.
>>> import random
>>> print(dir(random))
['BPF', 'LOG4', 'NV_MAGICCONST', 'RECIP_BPF', 'Random', 'SG_MAGICCONST',
'SystemRandom', 'TWOPI', '_BuiltinMethodType', '_MethodType', '_Sequence',
'_Set', '__all__', '__builtins__', '__cached__', '__doc__', '__file__',
'__loader__', '__name__', '__package__', '__spec__', '_acos', '_bisect', '_ceil',
'_cos', '_e', '_exp', '_inst', '_itertools', '_log', '_os', '_pi', '_random',
'_sha512', '_sin', '_sqrt', '_test', '_test_generator', '_urandom', '_warn',
'betavariate', 'choice', 'choices', 'expovariate', 'gammavariate', 'gauss',
'getrandbits', 'getstate', 'lognormvariate', 'normalvariate', 'paretovariate',
'randint', 'random', 'randrange', 'sample', 'seed', 'setstate', 'shuffle',
'triangular', 'uniform', 'vonmisesvariate', 'weibullvariate']
>>>     
  \end{Verbatim}

  \medskip
  
  \begin{itemize}
  \item Likewise: if you work a lot with arrays you will end up using mostly
    \texttt{numpy}
  \end{itemize}
\end{frame}


\begin{frame}
  \frametitle{Overview of the standard library}
  \framesubtitle{\texttt{os}, \texttt{os.path}, \texttt{glob} and
    \texttt{shutil}}
  \begin{itemize}
  \item Miscellaneous operating system interfaces
    \begin{itemize}
    \item Access filesystem (access, create and copy files and directories)
    \item List directory content
    \item Environmental variables
    \item Absolute and relative paths
    \item Exec OS commands
    \end{itemize}
  \item \alert{All of this in a cross-platform fashion}
  \end{itemize}
\end{frame}


\begin{frame}
  \frametitle{Overview of the standard library}
  \framesubtitle{\texttt{argparse}}
  \begin{itemize}
  \item \alert{Parser for command-line options---this is an important one!}
  \item Ever found yourself modifying the source code and running your
    program with different parameters?
    \begin{itemize}
    \item This is a terribly bad practice!
    \item And git will complain about modified files :-)
    \end{itemize}
  \item Keep the argparse documentation under your pillow!
  \end{itemize}
\end{frame}


\begin{frame}
  \frametitle{Overview of the standard library}
  \framesubtitle{\texttt{logging}}
  \begin{itemize}
  \item Ever found yourself inserting debug print() statements in the code
    when needed?
    \begin{itemize}
    \item This is another terrible bad practice!
    \item (In general you should )
    \end{itemize}
  \end{itemize}
\end{frame}


\begin{frame}
  \frametitle{References}
  \scriptsize
  \begin{itemize}
  \item \url{https://docs.python.org/3/library/}
  \item \url{https://pypi.org/}
  \item \url{https://docs.python.org/3/reference/import.html}
  \item \url{https://docs.python-guide.org/}
  \end{itemize}
\end{frame}



\end{document}
