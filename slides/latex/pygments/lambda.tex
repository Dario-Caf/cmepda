\begin{Verbatim}[label=\makebox{\url{https://github.com/lucabaldini/cmepda/tree/master/slides/latex/snippets/lambda.py}},commandchars=\\\{\}]
\PY{c+c1}{\PYZsh{} Here we create a lambda function and assign a name to it (ironically)}
\PY{n}{multiply} \PY{o}{=} \PY{k}{lambda} \PY{n}{x}\PY{p}{,} \PY{n}{y}\PY{p}{:} \PY{n}{x} \PY{o}{*} \PY{n}{y}
\PY{c+c1}{\PYZsh{} Use it}
\PY{k}{print}\PY{p}{(}\PY{n}{multiply}\PY{p}{(}\PY{l+m+mi}{5}\PY{p}{,} \PY{o}{\PYZhy{}}\PY{l+m+mi}{1}\PY{p}{)}\PY{p}{)}

\PY{c+c1}{\PYZsh{} Typical use is inside generator functions}
\PY{n}{numbers} \PY{o}{=} \PY{n+nb}{range}\PY{p}{(}\PY{l+m+mi}{10}\PY{p}{)}
\PY{n}{squares} \PY{o}{=} \PY{n+nb}{list}\PY{p}{(}\PY{n+nb}{map}\PY{p}{(}\PY{k}{lambda} \PY{n}{n}\PY{p}{:} \PY{n}{n}\PY{o}{*}\PY{o}{*}\PY{l+m+mi}{2}\PY{p}{,} \PY{n}{numbers}\PY{p}{)}\PY{p}{)}
\PY{k}{print}\PY{p}{(}\PY{n}{squares}\PY{p}{)}

\PY{c+c1}{\PYZsh{} However, remeber that you can do the same with list comprehension}
\PY{n}{squares} \PY{o}{=} \PY{p}{[}\PY{n}{n}\PY{o}{*}\PY{o}{*}\PY{l+m+mi}{2} \PY{k}{for} \PY{n}{n} \PY{o+ow}{in} \PY{n}{numbers}\PY{p}{]}
\PY{k}{print}\PY{p}{(}\PY{n}{squares}\PY{p}{)}

[Output]
-5
[0, 1, 4, 9, 16, 25, 36, 49, 64, 81]
[0, 1, 4, 9, 16, 25, 36, 49, 64, 81]
\end{Verbatim}